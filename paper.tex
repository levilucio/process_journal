%%%%%%%%%%%%%%%%%%%%%%% file template.tex %%%%%%%%%%%%%%%%%%%%%%%%%
%
% This is a general template file for the LaTeX package SVJour3
% for Springer journals.          Springer Heidelberg 2010/09/16
%
% Copy it to a new file with a new name and use it as the basis
% for your article. Delete % signs as needed.
%
% This template includes a few options for different layouts and
% content for various journals. Please consult a previous issue of
% your journal as needed.
%
%%%%%%%%%%%%%%%%%%%%%%%%%%%%%%%%%%%%%%%%%%%%%%%%%%%%%%%%%%%%%%%%%%%
%
% First comes an example EPS file -- just ignore it and
% proceed on the \documentclass line
% your LaTeX will extract the file if required
\begin{filecontents*}{example.eps}
%!PS-Adobe-3.0 EPSF-3.0
%%BoundingBox: 19 19 221 221
%%CreationDate: Mon Sep 29 1997
%%Creator: programmed by hand (JK)
%%EndComments
gsave
newpath
  20 20 moveto
  20 220 lineto
  220 220 lineto
  220 20 lineto
closepath
2 setlinewidth
gsave
  .4 setgray fill
grestore
stroke
grestore
\end{filecontents*}
%
\RequirePackage{fix-cm}
%
%\documentclass{svjour3}                     % onecolumn (standard format)
%\documentclass[smallcondensed]{svjour3}     % onecolumn (ditto)
\documentclass[smallextended]{svjour3}       % onecolumn (second format)
%\documentclass[twocolumn]{svjour3}          % twocolumn
%
\smartqed  % flush right qed marks, e.g. at end of proof
%
% \usepackage{mathptmx}      % use Times fonts if available on your TeX system
%
% insert here the call for the packages your document requires
%\usepackage{latexsym}
% etc.
%
% please place your own definitions here and don't use \def but
% \newcommand{}{}
%
% Insert the name of "your journal" with
% \journalname{myjournal}

\usepackage{calc}
\usepackage{amssymb}
\usepackage{amstext}
\usepackage{amsmath}
\usepackage{color, colortbl}

\usepackage{xcolor} 

\usepackage[final,pdftex]{graphicx}
        \pdfcompresslevel=9
        \DeclareGraphicsExtensions{.png .pdf} 

\usepackage{chngcntr}
\usepackage{epsfig} 
\usepackage{url}
 
\usepackage{ifthen} 
\usepackage{amssymb}
 
\usepackage{float} 

\newboolean{showcomments}
\setboolean{showcomments}{false} % toggle to show or hide comments
\ifthenelse{\boolean{showcomments}}
  {\newcommand{\nb}[2]{
    \fcolorbox{gray}{yellow}{\bfseries\sffamily\scriptsize#1}  
    {$\blacktriangleright$#2$\blacktriangleleft$}
   }
   \newcommand{\version}{\emph{\scriptsize$-$working$-$}}
  } 
  {\newcommand{\nb}[2]{}
   \newcommand{\version}{}
  } 

\usepackage[linewidth=1pt]{mdframed}
\usepackage{lipsum}  

% for comments
\newcommand\levi[1]{\nb{Levi}{\textcolor{teal}{#1}}}
\newcommand\moussa[1]{\nb{Moussa}{\textcolor{blue}{#1}}}
\newcommand\adrien[1]{\nb{Adrien}{\textcolor{red}{#1}}}

\newcommand \afthree {\textsc{AF3}}

\usepackage[font={small}]{caption, subfig}

% 
\begin{document}

\title{Process-Aware Model-Driven Development Environments}
%\subtitle{Do you have a subtitle?\\ If so, write it here}

%\titlerunning{Short form of title}        % if too long for running head

\author{First Author         \and
        Second Author %etc.
}

%\authorrunning{Short form of author list} % if too long for running head

\institute{F. Author \at
              first address \\
              Tel.: +123-45-678910\\
              Fax: +123-45-678910\\
              \email{fauthor@example.com}           %  \\
%             \emph{Present address:} of F. Author  %  if needed
           \and
           S. Author \at
              second address
}

\date{Received: date / Accepted: date}
% The correct dates will be entered by the editor


\maketitle

\begin{abstract}
Due to recent advances in Domain Specific Language (DSL) workbenches,
it has become possible to build model-driven development environments
as sets of individual DSLs that get composed for a specific purpose. In
this paper we explore how model-driven development environments can become
process-aware, to assist the user when building a model. We offer an
explanation to our ideas at three levels of abstraction: 1) the meta-meta level,
where brick DSLs are built using the Meta-Programming System (MPS) workbench;
2) the meta level, where brick DSLs are assembled into frameworks that
are further tailored for particular modelling scenarios through the introduction
of an explicit process for model construction; and 3) the model level, where
models are built through progressive tool-guided refinements and automated
steps based on the process introduced at the meta level. We exemplify our
approach by providing the main highlights of the ongoing development of a
model-driven requirements gathering environment for our industrial partners.
\levi{abstract from workshop paper, needs to be rewritten}
\keywords{First keyword \and Second keyword \and More}
% \PACS{PACS code1 \and PACS code2 \and more}
% \subclass{MSC code1 \and MSC code2 \and more}
\end{abstract}

\input{text/introduction}

Introduce the problem statement here, something along the following
lines: 

\begin{itemize}
  \item How do we customize a Model-Driven Development environment in order to
  support a given modelling process.
  \item The process should be of an advisory nature and minimally invasive
  regarding the modelling experience.
\end{itemize} 

\subsection{A conceptual frame for building Process-Aware Model-Driven
Development environments}
\label{sec:meta}
 

\section{Case Studies: The MPS and the AutoFOCUS MDD Environments}
\subsection{MPS and AF3}
\label{sec:mps}
\input{text/mps}

\subsection{Language stacks in MPS and AF3}

\subsection{Adding guidance to both environments}


\section{Aplication to practice}
%!TEX root = ../paper.tex

\subsection{\doone\ in \afthree}

The standard \doone~\cite{do178c} is the primary document by which the certification authorities approve all commercial software-based aerospace systems.
It requires a thorough definition and documentation of the software development process. 
The base set of required documentation and life cycle artifacts includes software requirements standard, software design standard and traceability.
\dothree\ supplements \doone\ with model-based development guidance.
The process described in this section guides the users (requirement engineer and architect) to generate models (requirements and their corresponding design) to fulfill some of the standards imposed by \doone\ and \dothree.

Good requirements standards should guarantee that the requirements information (ID, name, description, rationale, author, etc) is complete.
Also, HLRs should be traceable to SRs.
Once requirements are completed, they need to be reviewed (by someone which did not participate in their development).
According to \dothree, artifacts and communication channels should have meaningful names, ports should have a type and a range, etc.
\doone\ also imposes traceability between design and HLR.
This allows to guarantee e.g. that every HLR is implemented.

We show how the process described above can be implemented in \afthree\ using constraints.
The first objective is to add requirements to the project. 
Once this is done, two objectives become possible and can be performed in any order. 
\emph{Requirements information complete} assures that the ID, name, description, rationale, author and source (SR from where this requirement is derived) fields are completed. 
The latter defines a trace between HLRs and SRs.
\emph{Requirements have an unique aspect} guides the engineer to split a requirement when it focus on more than a particular aspect.
Consider a requirement `\emph{the value of $o_1$ should be computed as $i_1$ in less than time units}'.
This requirement contains a functional aspect (how to compute the value of $o_1$) and a timing aspect (how fast this should be computed).
We believe it is good practice to have both as separated requirements; the process guide the user to do so.

\subsection{IETS3 in MPS}
\label{sec:meta}
\input{text/iets3_in_mps}

\subsection{Assurance cases in AF3}
 

\section{Implementation}
\subsection{Implementation in MPS}

\subsection{Implementation in AutoFOCUS}

\subsection{Lessons learned}

\section{Discussion}

\section{Conclusion} 

\section*{Acknowledgements}
The work presented in this paper was developed \ldots

\bibliographystyle{abbrv}
\bibliography{bibliography}

\end{document}
% end of file template.tex

