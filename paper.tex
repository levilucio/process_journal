 %%%%%%%%%%%%%%%%%%%%%%% file template.tex %%%%%%%%%%%%%%%%%%%%%%%%%
%
% This is a general template file for the LaTeX package SVJour3
% for Springer journals.          Springer Heidelberg 2010/09/16
%
% Copy it to a new file with a new name and use it as the basis
% for your article. Delete % signs as needed.
%
% This template includes a few options for different layouts and
% content for various journals. Please consult a previous issue of
% your journal as needed.
%
%%%%%%%%%%%%%%%%%%%%%%%%%%%%%%%%%%%%%%%%%%%%%%%%%%%%%%%%%%%%%%%%%%%
%
% First comes an example EPS file -- just ignore it and
% proceed on the \documentclass line
% your LaTeX will extract the file if required
\begin{filecontents*}{example.eps}
%!PS-Adobe-3.0 EPSF-3.0
%%BoundingBox: 19 19 221 221
%%CreationDate: Mon Sep 29 1997
%%Creator: programmed by hand (JK)
%%EndComments
gsave
newpath
  20 20 moveto
  20 220 lineto
  220 220 lineto
  220 20 lineto
closepath
2 setlinewidth
gsave
  .4 setgray fill
grestore
stroke
grestore
\end{filecontents*}
%
\RequirePackage{fix-cm}
%
%\documentclass{svjour3}                     % onecolumn (standard format)
%\documentclass[smallcondensed]{svjour3}     % onecolumn (ditto)
\documentclass[smallextended]{svjour3}       % onecolumn (second format)
%\documentclass[twocolumn]{svjour3}          % twocolumn
%
\smartqed  % flush right qed marks, e.g. at end of proof
%
% \usepackage{mathptmx}      % use Times fonts if available on your TeX system
%
% insert here the call for the packages your document requires
%\usepackage{latexsym}
% etc.
%
% please place your own definitions here and don't use \def but
% \newcommand{}{}
%
% Insert the name of "your journal" with
% \journalname{myjournal}

\usepackage{calc}
\usepackage{amssymb}
\usepackage{amstext}
\usepackage{amsmath}
\usepackage{color, colortbl}

\usepackage{xcolor} 

\usepackage[final,pdftex]{graphicx}
        \pdfcompresslevel=9
        \DeclareGraphicsExtensions{.png .pdf} 

\usepackage{chngcntr}
\usepackage{epsfig} 
\usepackage{url}
 
\usepackage{ifthen} 
\usepackage{amssymb}
 
\usepackage{float} 

\newboolean{showcomments}
\setboolean{showcomments}{false} % toggle to show or hide comments
\ifthenelse{\boolean{showcomments}}
  {\newcommand{\nb}[2]{
    \fcolorbox{gray}{yellow}{\bfseries\sffamily\scriptsize#1}  
    {$\blacktriangleright$#2$\blacktriangleleft$}
   }
   \newcommand{\version}{\emph{\scriptsize$-$working$-$}}
  } 
  {\newcommand{\nb}[2]{}
   \newcommand{\version}{}
  } 

\usepackage[linewidth=1pt]{mdframed}
\usepackage{lipsum}  

% for comments
\newcommand\levi[1]{\nb{Levi}{\textcolor{teal}{#1}}}
\newcommand\moussa[1]{\nb{Moussa}{\textcolor{blue}{#1}}}
\newcommand\adrien[1]{\nb{Adrien}{\textcolor{red}{#1}}}

\newcommand \afthree {\textsc{AF3}}
\newcommand \doone {\textsc{DO-178C}}
\newcommand \dothree {\textsc{DO-331}}
\newcommand \asset {\textsc{ASSET}}

\usepackage[font={small}]{caption, subfig}

% 
\begin{document}

\title{Process-Aware Model-Driven Development Environments}
%\subtitle{Do you have a subtitle?\\ If so, write it here}

%\titlerunning{Short form of title}        % if too long for running head

\author{First Author         \and
        Second Author %etc.
}

%\authorrunning{Short form of author list} % if too long for running head

\institute{F. Author \at
              first address \\
              Tel.: +123-45-678910\\
              Fax: +123-45-678910\\
              \email{fauthor@example.com}           %  \\
%             \emph{Present address:} of F. Author  %  if needed
           \and
           S. Author \at
              second address
}

\date{Received: date / Accepted: date}
% The correct dates will be entered by the editor


\maketitle

\begin{abstract}
Due to recent advances in Domain Specific Language (DSL) workbenches,
it has become possible to build model-driven development environments
as sets of individual DSLs that get composed for a specific purpose. In
this paper we explore how model-driven development environments can become
process-aware, to assist the user when building a model. We offer an
explanation to our ideas at three levels of abstraction: 1) the meta-meta level,
where brick DSLs are built using the Meta-Programming System (MPS) workbench;
2) the meta level, where brick DSLs are assembled into frameworks that
are further tailored for particular modelling scenarios through the introduction
of an explicit process for model construction; and 3) the model level, where
models are built through progressive tool-guided refinements and automated
steps based on the process introduced at the meta level. We exemplify our
approach by providing the main highlights of the ongoing development of a
model-driven requirements gathering environment for our industrial partners.
\levi{abstract from workshop paper, needs to be rewritten}
\keywords{First keyword \and Second keyword \and More}
% \PACS{PACS code1 \and PACS code2 \and more}
% \subclass{MSC code1 \and MSC code2 \and more}
\end{abstract}

\section{Introduction}
\label{sec:intro}

The current trends in domain specific software engineering, domain specific
languages (DSL) and domain specific modelling languages (DSML) demonstrate the
interest for \emph{tailored} software solutions.
When it comes to model-driven engineering (MDE) tools, studies like \cite{DBLP:conf/models/WhittleHRBH13} 
show that this trend is justified: among the MDE tools considered
in the reported study, the ones which were successful in penetrating industry
were precisely those which were developed in a tailored manner for a given
audience, the recommendation being: ``Match tools to people, not the other way
around''.

Many technologies now precisely enable this sort of tailoring, among which 
JetBrains' MPS \cite{DBLP:conf/pppj/PechSV13},
Xtext \cite{DBLP:conf/oopsla/EysholdtB10},
Sirius \cite{DBLP:conf/asplos/HauswaldLZLRKDM15}, or the classic MetaEdit+ \cite{DBLP:conf/sle/Tolvanen16}.
With the maturity of these technologies, one can safely say that building your own MDE tool
has never been so easy.
This is an essential enabler and we can now clearly observe how technology
enthusiasts in various industries put this opportunity to good use, developping their own domain- or even company-specific tools.

On the other hand, as \cite{DBLP:conf/models/WhittleHRBH13} also mentions, tools are an enabler,
but they are not everything: ``More focus on processes, less on tools".
Even when a tailored tool is available, allowing the modelling of one's domain through
many sub-DSLs makes it such that new users often overwhelmed by the amount
of modelling techniques at their disposal. This is in fact the case for even
very specialized tools like Sfit \cite{DBLP:conf/vamos/BayhaLAMI16} for the
modelling of industrial manufacturing -- while modelling a restricted domain,
the tool contains a large number of different models, mostly rendered as
diagrams.
The question of methodology or process then naturally follows: in which order
should one use the diagrams? More generally, which information should one model
at a given point in time? The funding of several research projects precisely
focusing on this question, in particular in connection with MDE, demonstrate the relevance of this
question:
CESAR \cite{CESAR}, SPES \cite{DBLP:books/sp/pohl12}, SPES-XT \cite{DBLP:books/sp/spes2016},
all target the development of methodologies for the domain of embedded systems.
Similarly, Arcadia \cite{DBLP:conf/syscon/BonnetVEN16} emphasizes the importance
of the methodology in connection with MDE.

Just like for tools however, methodologies and processes can seldom be general enough to match 
all use cases and answer all needs. There is henceforth also a need for tailoring at
that level. To facilitate the acceptance of the methodology, it is however also essential
that the tool \emph{supports} the methodology: this is for instance the case
with Capella and Arcadia.
Capella however, supports only the Arcadia methodology which
is specific to avionic systems engineering and to the processes of Thales.
All the above points to the fact that, if the tools can be tailored, the process
itself should be tailorable.

In this paper we propose a framework to add process awareness to model-driven
development environments. We first define the framework by introducing its
conceptual bricks: the \emph{constraints}, the \emph{objectives} and the
\emph{process}. Constraints are predicates that evaluate certain properties of a
model. Constraints are grouped into \emph{objectives}, which constitute tasks
that should be achieved by the modeller. The \emph{process} provides the means
to order such objectives in terms of their inter-dependencies.

We then go on to instanciate this framework in two concrete model-driven
development environments: MPS and AF3.

In order to clarify our work we draw an analogy with the M-levels defined by the Object Management Group:
\begin{itemize}
  \item M3: the MPS tool with its language definition capabilities.
  \item M2: development and composition of ``brick DSLs'' in a
  domain-specific model-driven development environment, together with a
  model construction process for that environment.
  \item M1: usage of the developed domain-specific tool.
\end{itemize}
In addition, we identify four different \emph{roles} in the development and
usage of the framework:
\begin{itemize}
  \item the \emph{framework developer} (in our case JetBrains), who develop MPS
  (level M3),
  \item the \emph{framework customizer} (the authors of this paper), who develop
  a library for process-customizable DSLs (level M2),
  \item the \emph{(domain-specific) tool developer} 
    (typically a consultant or the in-house technology department of a company)
    who actually develops the domain-specific tool, making use of our libraries (level M2),
  \item and the \emph{user} (level M1).
\end{itemize}

In this work, we contribute a framework at level M2 to support the tool developer
in developing a \emph{process-aware} domain-specific tool.
In particular we have implemented a so-called \textsf{Process} language for
describing a set of refinement steps including descriptions which are then used to guide and help the user.
At the M1 level, a ``dashboard'' allows the user to know permanently the next step to achieve.

% The remainder of this paper is structured as follows.  In
% section~\ref{sec:metameta} we describe the MPS framework, which we use as the
% technological basis for all the results presented in this paper.
% Section~\ref{sec:meta} then describes our case study -- a set of languages and a
% process for the incremental gathering and refinement of requirements,
% specialized for hardware cooling systems. Then, in section~\ref{sec:model},
% we exemplify the construction of the requirements for a specific fan-based cooling system, using
% the previously defined languages and refinement process.
% Section~\ref{sec:implementation} lifts the veil over some of the implementation
% details of our work and section~\ref{sec:related_work} provides pointers to work
% in the literature that closely relates to the results we present here. Finally,
% section~\ref{sec:conclusion} presents a discussion of this research and 
% potential future work.

% The remainder of this paper is structured as follows.  In
% section~\ref{sec:metameta} we describe the MPS framework, which we use as the
% technological basis for all the results presented in this paper.
% Section~\ref{sec:meta} then describes our case study -- the construction of a
% model-driven development environment for the incremental gathering and
% refinement of requirements. Then, in section~\ref{sec:model}, we exemplify the
% construction of the requirements using the environment described in the previous
% section. Section~\ref{sec:implementation} lifts the veil over some of the
% implementation details of our work and section~\ref{sec:related_work} provides
% pointers to work in the literature that closely relates to the results we
% present here. Finally, section~\ref{sec:conclusion} presents a discussion of
% our research and potential future work.



Introduce the problem statement here, something along the following
lines: 

\begin{itemize}
  \item How do we customize a Model-Driven Development environment in order to
  support a given modelling process.
  \item The process should be of an advisory nature and minimally invasive
  regarding the modelling experience.
\end{itemize} 

\section{A conceptual frame for building Process-Aware Model-Driven
Development environments}
\label{sec:meta}
\subsection{A conceptual frame for building Process-Aware Model-Driven
Development environments}
\label{sec:meta}
 

\section{Case Studies: The MPS and the AutoFOCUS MDD Environments}
\subsection{MPS and AF3}
\label{sec:mps}
The metameta level (M3, in MOF terms) is where the bricks for our approach are
built. These bricks consist of Domain Specific Languages, defined in the MPS
(Meta Programming System)~\cite{mps} framework. MPS is a stable and
industrially-proven projectional meta-editor. Being a meta-editor, MPS provides edition
capabilities at the meta-levels we need for our approach, in particular M2 and
M1. It uniformly integrates language and editor design capabilities, together
with code generation tools and in-built correct-by-construction tactics such as meta-model
conformance, syntax highlighting, auto-completion or type checking.  
MPS is developed by JetBrains, which assumes the role of \emph{framework
developer}. 

Throughout this paper we will often use vocabulary that is close to that used in
the MPS world in order to remain aligned with the technical aspects of our
work. In particular, the following terms are recurrently used in what follows:

\begin{itemize}
  \item \emph{Language}: an MPS language includes a metamodel, in the classical
  EMF sense. It additionally includes one or more editors for its metamodel,
  which provide concrete syntax. Other aspects of a language
  can be defined and custom new aspects can be
  built by the MPS user (see~\cite{mps} for details).
  \item \emph{Solution}: MPS solutions are projects where users can import
  MPS languages and create their models using those languages.
  \item \emph{Concept}: the MPS equivalent of metamodel class.
  \item \emph{Concept / language instance}: concepts can be instantiated, in
  the same way metamodel classes can. We will also sometimes write
  \emph{language instance} to refer to an instance of the \emph{root} concept of
  an MPS language.
  \item \emph{Intentions}: arbitrary actions attached to concepts of a
  language. Those actions can be launched by the user when the
  focus of the editor is on objects which are instances of those concepts.
  \item \textsf{BaseLanguage Java}: most of MPS' complex language operations are
  coded using the predefined \textsf{BaseLanguage} MPS language, a projectional
  replica of Java enriched with MPS-specific constructs.
  \item \emph{Language composition}:  Reference or containment relations can exist between instances of
  concepts of different languages, which is the primary language composition
  mechanism in MPS. Additionally, an MPS model can contain instances of concepts
  belonging to many languages (not necessarily referring to each other), which
  provides an additional means for language composition.
\end{itemize}

% At this level we had to extend the existing MPS framework with languages to
% define \emph{flow} and a \emph{dashboard}. Additionally, we have provided
% extension points to give the user the possibility to define her own constraints
% that can be used to direct the flow of edition of the composed model.


\subsection{Language stacks in MPS and AF3}

\subsection{Adding guidance to both environments}


\section{Aplication to practice}
%!TEX root = ../paper.tex

\subsection{\doone\ in \afthree}

The \doone\ standard~\cite{do178c} is the primary document by which the certification authorities approve all commercial software-based aerospace systems.
It requires a thorough definition and documentation of the software development process. 
The base set of required documentation includes requirements standard, design standard and traceability.
The \dothree\ standard supplements \doone\ with model-based development guidance.
Both standards define what needs to be done (the objectives that the project must fulfill), but they do not specify how to do so (they do not define a process) .
It is up to the process engineer of each project to define the actual process and document how objectives are achieved.
In this section we describe one such a process and show how it can be implemented in \afthree.
The process guides the requirement engineer and architect to generate requirements and their corresponding design to fulfill some of the standards imposed by \doone\ and \dothree.

Requirements standards required by \doone\ normally impose that information about requirements (ID, name, description, rationale, author, etc) is complete.
Traceability requires that every HLR is related to a SRs.
Once requirements are completed, they need to be reviewed and approved by someone which did not participate in their development. 
The latter guarantees verification independence~\cite{cast26} which must be clearly documented.
According to \dothree, artifacts and their communication channels should have meaningful names, ports should have a type and a range, etc.
Traceability also requires that every artifact in the design shall be traced to a HLR.

We show how the objectives mentioned above can be achieved following our proposed development process.
Objectives are implemented in \afthree\ using constraints.
The first objective is to add requirements to the project. 
Once this is done, two objectives become possible and can be performed in any order. 
\emph{Requirements information complete} assures that the ID, name, description, rationale, author and source (SR from where this requirement is derived) fields are completed. 
The latter defines traces between HLRs and SRs.
\emph{Requirements have an unique aspect} guides the engineer to refine a requirement when it focus on more than a particular aspect.
Consider the following requirement: `\emph{the value of output $o_1$ is the value of input $i_1$ plus 1 and should be computed in less than 5 time units}'.
This requirement contains a functional aspect (how to compute the value of $o_1$) and a timing aspect (what the performance of the computation should be).
Since they refer to different aspects, it is good practice to have both as separated requirements.
The proposed process guides the user to do so.

\subsection{IETS3 in MPS}

\subsection{Assurance cases in AF3}
 

\section{Implementation}
\subsection{Implementation in MPS}

\subsection{Implementation in AutoFOCUS}

\subsection{Lessons learned}

\section{Discussion}

\section{Conclusion} 

\section*{Acknowledgements}
The work presented in this paper was developed \ldots

\bibliographystyle{abbrv}
\bibliography{bibliography}

\end{document}
% end of file template.tex

