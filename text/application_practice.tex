%!TEX root = ../paper.tex

\subsection{\doone\ in \afthree}

The \doone\ standard~\cite{do178c} is the primary document by which the certification authorities approve all commercial software-based aerospace systems.
It requires a thorough definition and documentation of the software development process. 
The base set of required documentation includes requirements standard, design standard and traceability.
The \dothree\ standard supplements \doone\ with model-based development guidance.
Both standards define what needs to be done (the objectives that the project must fulfill), but they do not specify how to do so (they do not define a process) .
It is up to the process engineer of each project to define the actual process and document how objectives are achieved.
In this section we describe one such a process and show how it can be implemented in \afthree.
The process guides the requirement engineer and architect to generate requirements and their corresponding design to fulfill some of the standards imposed by \doone\ and \dothree.

Requirements standards required by \doone\ normally impose that information about requirements (ID, name, description, rationale, author, etc) is complete.
Traceability requires that every HLR is related to a SRs.
Once requirements are completed, they need to be reviewed and approved by someone which did not participate in their development. 
The latter guarantees verification independence~\cite{cast26} which must be clearly documented.
According to \dothree, artifacts and their communication channels should have meaningful names, ports should have a type and a range, etc.
Traceability also requires that every artifact in the design shall be traced to a HLR.

We show how the objectives mentioned above can be achieved following our proposed development process.
Objectives are implemented in \afthree\ using constraints.
The first objective is to add requirements to the project. 
Once this is done, two objectives become possible and can be performed in any order. 
\emph{Requirements information complete} assures that the ID, name, description, rationale, author and source (SR from where this requirement is derived) fields are completed. 
The latter defines traces between HLRs and SRs.
\emph{Requirements have an unique aspect} guides the engineer to refine a requirement when it focus on more than a particular aspect.
Consider the following requirement: `\emph{the value of output $o_1$ is the value of input $i_1$ plus 1 and should be computed in less than 5 time units}'.
This requirement contains a functional aspect (how to compute the value of $o_1$) and a timing aspect (what the performance of the computation should be).
Since they refer to different aspects, it is good practice to have both as separated requirements.
The proposed process guides the user to do so.

\subsection{IETS3 in MPS}

\subsection{Assurance cases in AF3}
